
Our technique follows a general methodology for designing
systems supporting programming by examples~\cite{Harris:2011}.
It includes the following three major steps:
(1) identify a language subset (of SQL) that is expressive to
describe a large proportion of %operations that users need to perform
database queries; (2) design an algorithm that
can efficiently infer programs in the language from input-output
examples; and (3) develop a ranking strategy that presents
the most likely program to the users, when multiple solutions exist.

\vspace{1mm}
\noindent \textbf{\textit{Supported SQL Subset.}}
The full SQL 93 specification contains over 1000 pages, and it is
infeasible and unnecessary to support all language features in practice.
When selecting a supported SQL subset, there is a tradeoff
between the expressiveness of the selected language subset,
and the complexity of finding simple consistent solutions
within that language's search space.  In general, the more expressive a search
space, the harder the task of finding consistent hypotheses within
that search space. 

During our study of end-user's real problems, we also carefully studied what would
be the solution to users' desired query, and found that the most common
desired queries can be composed by a handful of basic SQL constructs,
which form the target language of this work (Section~\ref{sec:langsubset}).

As shown in our experiments (Section~\ref{sec:evaluation}), the supported subset, though far from completion, is expressive
enough to describe many query tasks succinctly, while at the same time concise
enough to be amenable for efficient inference. 



\vspace{1mm}
\noindent \textbf{\textit{SQL Query Inference Algorithm.}} Our inference
algorithm, as described in Section~\ref{sec:approach}, consists of two
major steps. The first step, called \textit{SQL Skeleton Creation}, scans
the example input and output, and \textit{guesses} what a desirable
SQL query may look like. The output of this step is a set of \textit{incomplete} SQL
query skeletons. A SQL query skeleton is an incomplete SQL query,
which contains unknown parts. The second step,
called \textit{SQL Query Completion} takes as inputs a created
SQL skeleton, fills unknown parts with possible solutions, and then produces a set of
syntactically-valid SQL queries. In particular, when performing
SQL query completion, our technique uses a rule learning algorithm
to infer the conditions (Section~\ref{sec:decision_tree}), and uses 
type-directed search to infer possible aggregates in a query (Section~\ref{sec:agg_search}).

\vspace{1mm}
\noindent \textbf{\textit{Query Candidate Ranking Strategy.}}
Input-output examples provided by end-users are likely to
be under-specified. It is entirely possible that multiple
queries can be synthesized to satisfy the provided examples.
We address this problem by developing a ranking scheme that 
ranks the possibly multiple queries
consistent with the given input-output
examples. Our ranking scheme, detailed in Section~\ref{sec:ranking}
is inspired by the Occam's razor principle, prefers a smaller and
simpler solution if other aspects being equal.



Figure~\ref{fig:workflow} sketches the general workflow of our technique.
At a high level, our technique consists of three major steps: Query Skeleton
Creation(Section~\ref{sec:skeleton}), SQL Query Completion
(Section~\ref{sec:completion}), and Query Candidate Ranking (Section~\ref{sec:ranking}).
Specifically, our approach takes from users input-output examples. It first infers
a partially-complete query skeleton. The inferred query skeleton, though contains
incomplete parts that can not be yet decided, serves as a good reference
for further synthesizing a complete SQL query.
After that, our technique uses two techniques to complete a SQL skeleton.
In particular, it uses an advanced rule learning algorithm from the machine
learning community to infer query conditions (Section~\ref{sec:decision_tree}), and
 employs type-directed search to figure out possible aggregation
expressions (Section~\ref{sec:agg_search}).
The SQL Query Completion step produces a list of syntactically-valid
query candidates that satisfy the provided input-output examples.
However, it is possible that multiple query candidates can be synthesized based on the
provided input-output example. To deal with that, our technique

%This makes our tool more usable.

