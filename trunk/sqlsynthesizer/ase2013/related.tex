
\section{Related Work}
\label{sec:related}

This section discusses two categories
of closely-related work on reverse query processing 
and automated program synthesis.


\subsection{Reverse Query Processing}

In the database community, there is a rich body of
work~\cite{Zloof:1975, Tran:2009, DasSarma:2010} that share the same broad principle
of ``reverse query processing''. Zloof's work on \textit{Query by Example} (QBE)
~\cite{Zloof:1975} provided an intuitive form-based Graphical User Interface for
writing database queries, which is completely different than ours.
\todo{add why it is different}



Tran et al.~\cite{Tran:2009} studied the problem of \textit{Query By Output} (QBO). 
The problem of QBO is similar to the view synthesis problem, but the technique
presented in~\cite{Tran:2009} infers simple select-project-join queries.
Of the examples presented in this paper, such queries can only be applied
to satisfy the example in \todo{some example}, and such queries cannot
be applied to many of the other real-world cases that we studied (Section~\ref{}).
The key limitation of \todo{fail to encode extra features}


Recently, Sarma et al.~\cite{DasSarma:2010} proposed a technique
to address the \textit{View Definitions Problem} (VDP).
\todo{explain more here}
With a rather different goal than \ourtool,
VDP aims to to find the most succinct and accurate view definition, when
the view query is restricted to a specific family of queries.
VDP can be treated as a special
case of QBO where $n = 1$ and there are no joins nor projections.
Compared to \ourtool, there are three key differences:
(1) VDP aims to infer a relation from a large representative
example view, while \ourtool attempts to infer a SQL query from a
set of few input-output examples \todo{revise below}
(which is a critical usability aspect for end-users who lack adequate
SQL experience). (2) the technique proposed by Sarma et al.~\cite{DasSarma:2010} does
not consider join or projection
operations and assumes the output view to be a strict subset of the input
database with the same schema\todo{revise}, while our work eliminates these
assumptions, and considers both join and projection operations.\todo{revise above}



\todo{what is the limitation of non program sythesis, why
sythensis would be useful.}

\subsection{Automated Program Synthesis }


Program synthesis~\cite{Gulwani:2010:DPS} aims to create an executable program
in some underlying language from specifications that can range
from logical declarative specifications to examples or
demonstrations~\cite{Harris:2011, singh:2012, Gulwani:2011,
Kandel:2011, Fisher08Pads,Lau:2003:PDU, Lau:2000:VSA, Barbosa:2010:MLA, Arasu:2009:LST}.

Automated program synthesis techniques 
has been used recently for many applications
such as synthesis of efficient low-level code~\cite{Solar-Lezama:2005},
data structure manipulations~\cite{Fisher:2008},
geometry constructions~\cite{Gulwani:2011:SGC},
snippets of excel macros~\cite{Harris:2011},
relational data representations~\cite{Barbosa:2010:MLA, Arasu:2009:LST} and string
expressions~\cite{singh:2012, Gulwani:2011}.


PADS~\cite{Fisher:2008} takes a large sample of unstructured data and infers a
format that describes the data. Users then manually define tools for
data in the format.
Though well-suited for their intended tasks, these
systems are insufficient for the database query tasks.
For example, while text extraction and mass editing
are quite valuable for data transformation, the above tools
lack other needed operations such as table
re-shaping, aggregation, and missing value imputation.
\todo{explain}


Closely related to our current work is Harris and Gulwani�s
system for learning excel spreadsheet transformations
from an example input-output pair~\cite{Harris:2011}. Given input and output
tables, their system can infer an excel macro that filters,
duplicates, and/or reorganizes table cells to generate the output table.
These resulting programs can express a variety of table
re-shaping operations. \ourtool differs in multiple respects.
First, Harris and Gulwani's approach treats table cells
as atomic units, and thus has limited expressiveness.
For example, it does not support queries involving text editing or extraction\todo{xxx}.
Second, \todo{explain the difference between excel work}


The Wrangler tool, developed in the HCI community, provides a nice visual
programming-by-demonstration interface to table transformations
for data cleaning~\cite{Kandel:2011}. By contrast, \ourtool provides
an interface based on examples, which is more
friendly to end users. \todo{xxx}


There is some recent work for inferring SQL queries for database end-users.
Query recommendation systems, such as SQLShare~\cite{} and SQLSnippet~\cite{},
\todo{explain}
may achieve a similar goal of helping end-users write better SQL
queries, but they rely on information
that we cannot assume access to in many realistic scenarios: a query log~\cite{Khoussainova:2010},
or user history and preferences~\cite{Howe:2011}.

Cheung et al~\cite{abs-1208-2013} presented a technique to infer SQL
queries from imperative code.
Their technique automatically identifies
fragments of application logic (written in an imperative language
like Java) that can be pushed into
SQL queries. 
\todo{focus on performance...}

%also cite this~\cite{Cheung:2012:UPS}, combining machine learning with program synthesis.
