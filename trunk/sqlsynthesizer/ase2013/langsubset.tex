\newcommand{\q}{\langle query\rangle}
\newcommand{\db}{\langle db\rangle}
\newcommand{\pat}{\langle pat\rangle}
\newcommand{\bug}{\langle bug\rangle}
\newcommand{\dist}{\langle distance\rangle}
\newcommand{\sem}[1]{\llbracket #1\rrbracket}
\newcommand{\lit}[1]{\texttt{#1}}

\newcommand{\column}{\langle column\rangle}
\newcommand{\dbtable}{\langle table\rangle}
\newcommand{\cond}{\langle cond\rangle}
\newcommand{\op}{\langle op\rangle}
\newcommand{\e}{\langle expr\rangle}
\newcommand{\ce}{\langle cexpr\rangle}

\begin{figure}[t]
%\scriptsize{%
\footnotesize%
\begin{align*}
\q ::= {} 
	& \texttt{ SELECT } \e^+ \texttt{ FROM } \dbtable^+ \\
        & \texttt{ WHERE } \cond^+ \\ 
	&  \texttt{ GROUP BY } \column^+ \texttt{ HAVING } \cond^+\\
\dbtable::= {} &\ atom \\
\column ::= {} &\ \dbtable.atom\\
\cond ::= {} &\ \ \cond \;\texttt{\&\&}\; \cond \\ 
    & |\ \cond \;\texttt{||}\; \cond \\
    & |\ \texttt{(}\;\cond\;\texttt{)} \\
    & |\ \ce \;\op\; \ce \\
    & |\ \texttt{NOT EXIST} \;(\q) \\
\op ::= {} &\ \ \texttt{=} \;\;|\;\; \texttt{>}  \;\;|\;\; \texttt{<}\\
\ce ::= {} &\ \ const \;\;|\;\; \column  \;\; \\
\e ::= {} & \ce \;\;|\ count(\column) \\
    & |\ sum(\column) \;\;|\ max(\column) \;\;|\ min(\column) 
\end{align*}
\normalsize%
\caption{Syntax of the supported SQL subset.}
\label{fig:syntax}
\end{figure}


\section{Language and Example}
\label{sec:langsubset}

In this section, we first present the supported SQL subset, and
then describe a motivating example which could be expressed in the
supported SQL subset.

\subsection{Supported SQL Subset}

Figure~\ref{fig:syntax} defines the syntax of the supported language,
which
is a subset of the standard SQL 93 language. This language subset
supports common query operations across multiple tables (i.e.,
\CodeIn{select} ... \CodeIn{from} ... \CodeIn{where}..), and 
conjunction of predicates. The language
subset shares the same semantics with the standard SQL language.
Differing from language subsets used in the
existing work~\cite{DasSarma:2010}, our language subset
supporting
joining operations across multiple tables, and includes
widely-used database operations such as \CodeIn{group by},
\CodeIn{order by}, and \CodeIn{having}, as
well as a few common aggregation functions such as \CodeIn{count}, \CodeIn{sum},
\CodeIn{max}, and \CodeIn{min}.

As we will show in Section~\ref{sec:evaluation}, this
subset, though far from completion, is expressive enough for answering
many evaluated SQL exercises from a textbook and some real problems
from online forums.

