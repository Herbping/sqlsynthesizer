\subsection{Query Skeleton Creation}
\label{sec:skeleton}

In this step, our algorithm scans the given input-output example, aligns the output table data
with the input table data, and infers a query skeleton that captures the basic structure
of a desirable SQL query. The inferred query skeleton is a partial, incomplete SQL query, and
contains possible language structure that can be yet decided.
%, and serves as a good
%reference for further query completion in the next step.

When the algorithm scans the input-output examples, it uses the table \textit{column name}
as the major indicator to establish the input and output table relations. It uses a technique called
\textit{template-based} program synthesis~\cite{sttt-synthesis} to construct a query skeleton as follows:


\begin{itemize}

\item \textit{\textbf{Step 1: Determining Table Set.}} If multiple tables are provided as input, it is very likely that all
provided tables will be used in constructing the desirable SQL query. Based on this observation, we assume that
the minimum number of used tables is the number of all tables provided in the input example. By default, the
table set $T$ used in SQL query contains all given tables. However, it is
possible that a single table will be used in one SQL query for more than once, leading to a more complex query.
Our program does not forbid this case, rather, we \textit{prefer} to use as a few tables as possible to achieve
the same results. In some cases, if one field from a particular input table appears more than once in the
output table, we view it as a strong indicator that this table will be joined multiple times and add it to our table
set $T$ using an alias.


\item \textit{\textbf{Step 2: Determining Joining Columns. }} Given two arbitrary tables, there exist many
ways to join them. Enumerating all possible joining conditions would introduce a huge search space and thus
become intractable. We observe that, in practice, two tables are often joined \textit{only} via the following
three cases: first, tables are joined on their primary keys that have the same (or compatible) data types (i.e., it does not
make any sense to join two tables on a Integer column and a String column). Second, tables are joined
using the columns with the same column name, such as joining a \textit{Student} table with a \textit{Course} table on the
\textit{student\_name} column. Third, two columns that have the data type, and have a large portion of
overlapped data value in the given input can be used as a joining condition. In our tool implementation, it is trivial to check the first 
two cases. For the third case, our algorithm scans the whole given input table to check the ``value similarity''
between two arbitrary columns, and selects columns whose ``value similarity'' is above a fixed threshold as joining candidates.

\item \textit{\textbf{Step 3: Determining Output Columns.}} After determining the table set and joining columns,
the next step is to identify potential column names on which the result would be projected. If a
column in the output table  does not appear in any input table's column list, this output column must
be produced by aggregation. Our algorithm keeps track of these columns and appends a \CodeIn{group by} ... \CodeIn{having} ...
clause to the query skeleton.

\end{itemize}

In summary, this step infers two parts as a query skeleton: a table set used in producing the SQL query, a set of joining conditions
used to connect tables from the table set, and a list of columns used to project the output data.
It is worth noting that the results obtained from the above steps are not \textit{safe} in
terms that they may miss some valid SQL queries. We made the above assumption for the sake of tractability,
since in theory, the bound of table number in a SQL query is $O(n_t!)$, where $n_t$ is the number of given tables;
while the bound of possible number of join is $O(c_t^2)$ and the bound of the number of conditions is $O(n_t!n_tc_t^2)<O(n_t^3c_t^2)$.



