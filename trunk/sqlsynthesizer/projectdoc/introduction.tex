\section{Introduction}
\label{sec:introduction}


The big data revolution over the past few years has resulted
in significant advances in digitization of massive amounts
of data and accessibility of computational devices to massive
proportions of the population. A perennial challenge faced by many
enterprise nowadays is the management
of their increasing large and complex databases, which can contain
hundreds and even thousands of tables. 

The problem is exacerbated by the fact that many end-users
have myriad diverse backgrounds including research scientists,
business analysts, commodity traders, human resource managers,
finance professionals, and marketing managers. Increasingly,
those end-users who need to query databases to analyze the
data are not professional programmers, but are experts in some
other domains. They need to ask a variety of questions on their
data and use the answer to support their business decisions.

%but they often need to
%query databases to obtain relevant data to support their business decisions.


\subsection{Problem Statement}

Although the relational database management system (RDBMS) and the
de facto language (SQL) are perfectly adequate for many end-users'
needs~\cite{Howe:2011}, the costs associated with deployment and
use of database software and SQL are prohibitive. For example, as pointed out
by~\cite{Gray:2005}, conventional RDBMS software remains underused
in the long tail of science: the large number of users, such as the
research scientists who are in relatively small labs and individual
researchers, have limited IT funding, staff and infrastructure yet
collectively produce the bulk of scientific knowledge. Similarly, other end-users
such as business analysts who query databases frequently often
do not have significant SQL expertise.

For a large number of end-users, they use SQL queries in
the following typical ways: they browse textbook or online resources to learn basic
idioms of SQL. They try to write small examples and execute them
on real databases to explore the data itself. They modify
them to derive new queries by adding or removing snippets:
predicates in the where clause,
tables in the from clause, columns in the select clauses.
However, such practice is inefficient and time-consuming.
A key challenge is that many end-users can clearly understand
their goals but often can not easily get a correct SQL query for
their tasks, either due to the syntax complexity of the SQL language,
or the structure complexity of the underlying database tables, or others.
For most end-users, they need a ``computer'' to assist them to perform
complex and critical tasks. A highly accessible tool
that end-users can use to connect their intentions to executable
SQL queries would be best.


\begin{figure}[t]

\textbf{The input Table1:}

\begin{tabular}{|c|c|c|c|}
\hline
 Column1 & Column2& Column3& Column4\\
 \hline
 101 &  2001 &  3020 &  01-01-11\\
\hline
 101 &  2001 &  3002 &  02-01-11\\
\hline
 101 &  2001 &  3001 &  03-01-11\\
\hline
 102 &  2002 &  3002 &  01-01-11\\
\hline
\end{tabular}

\vspace{2mm}

\textbf{The input Table2:}

\begin{tabular}{|c|c|c|}
\hline
 Column1 & Column2& Column3 \\
 \hline
 20011 &  2001 &  200131\\
\hline
 20012 &  2001 &  200132\\
\hline
 20013 &  2001 &  200133\\
\hline
\end{tabular}

\vspace{2mm}

\textbf{The input Table3:}

\begin{tabular}{|c|c|}
\hline
 Column1 & Column2\\
 \hline
 20011 &  Site\\
\hline
 20012 &  Site\\
\hline
 20013 &  Site\\
\hline
\end{tabular}

\vspace{3mm}

\textbf{The output table:}

\begin{tabular}{|c|c|c|c|}
\hline
 101 & 200131 & 01-01-11 & Site\\
 \hline
 101 & 200132 & 01-01-11 & Site\\
 \hline
 101 & 200133 & 01-01-11 & Site\\
 \hline
\end{tabular}
\Caption{{\label{fig:example} Input-output example tables from
an online SQL help forum thread. Our approach is expected to
automatically infer a SQL query as shown in Figure~\ref{fig:answer}
that produces the output table for the three input tables.
}} \vspace{2mm}
\end{figure}

\subsection{Motivating Example}

We start out by describing a real-world example picked up from
an online SQL help forum, that illustrates a typical scenario
in writing SQL queries\footnote{\url{http://forums.tutorialized.com/sql-basics-113/join-problem-147856.html}}.

The thread was started by a novice user, who needed help to write a
SQL query to get result from three input tables. In the help thread, the
novice user described his required query in a few paragraphs of
English, but also include several small, representative input-output
examples as shown in Figure~\ref{fig:example}, to better express
his intention. 

The desirable SQL query, as shown in Figure~\ref{fig:answer} is
unintuitive to write. It first needs to join three input tables
on columns \CodeIn{T1.Column2}, \CodeIn{T2.Column2},
\CodeIn{T2.Column1}, and \CodeIn{T3.Column1},
then filters out tuples whose \CodeIn{Table1.Column2} value
is not '2001'. Further, it aggregates the results based on
the value of \CodeIn{Table2.Column3}, and returns the minimal
values of columns \CodeIn{T1.Column1}, \CodeIn{T1.Column4}, and \CodeIn{T3.Column2}
in each aggregated group as the results.

Such help threads on online discussion forums are not rare, but the SQL
queries needed to address them are non-trivial.


\subsection{Existing Solutions}

The state-of-the-art approaches in helping end-user writing SQL
queries is far from satisfactory. This is largely because 
a database management systems often come with tons
of features, but end-users struggle to find the correct feature or
succession of commands to use from a maze of features to accomplish
their tasks.

\textit{Graphical User Interface} (GUI) and \textit{general programming languages}
are two state-of-the-art approaches to reduce end-users' burden.
Many RDBMS come with a well-designed GUI.
However, a GUI often do not permit users to personalize
a database's functionality for user-specific tasks. On the other hand,
as a GUI supports more and more customization features, users
may struggle to discover those features, which can significantly
degrade its usability. General programming languages, such as SQL
or Java (with JDBC), serve as a fully expressive medium  for
communicating a user's intention to a database. However, learning
a practical programming language often requires a substantial amount
of time and energy that a typical end-user would not prefer,
and should not be expected, to invest. 



\begin{figure}[t]
\begin{CodeOut}
\begin{alltt}
\textbf{select min(T1.Column1), T2.Column3,
       min(T1.Column4), min(T3.Column2)
from Table1 T1, Table2 T2, Table3 T3
where T1.Column2 = T2.Column2 and T2.Column1 = T3.Column1
       and T2.Column2 = '2001'
group by T2.Column3}
\end{alltt}
\end{CodeOut}
\vspace{-5mm}
\Caption{{\label{fig:answer} The expected SQL query inferred
by our approach for the input-output examples in Figure~\ref{fig:example}.}}
\end{figure}


\subsection{Proposed Approach}

After carefully studying how end-users were describing their encountered
SQL query problems on online help forums, we observed that
one of the most common ways for end-users to
express their intents is using input-output examples. Although
input-output examples may lead to underspecification, they
serve as a good starting point and the natural interface
that a tool can provide assistance.


Inspired by this observation, in this work, we present an approach for synthesizing a SQL query
from input-output examples. Our approach takes an example input
table(s) and output table and automatically infers a SQL query that
implements a transformation that satisfies the example. If the
SQL query is applied to the
example input, then it produces the example output, and if the
SQL query is applied to other similar tables (potentially much larger),
then SQL query produces a corresponding output.

Our approach follows a general methodology for designing
systems supporting programming by examples~\cite{Harris:2011}.
It includes the following steps:
(1) identify a language subset (of SQL) expressive enough to
perform database query tasks 
on which a large class of users are struggling. (2) design a
programming language representation that can describe a large
proportion of operations that users need to perform on
relational database tables. (3) design an algorithm that
can efficiently infers programs in the language from input-output
examples. and (4) translate the program to the target
language, i.e., SQL.


\subsubsection{Goal}

We plan to develop a  SQL query synthesis tool that is capable of
synthesizing a wide range of SQL queries from input-output examples.
The synthesizer aims to replace the role of the forum expert,
which not only removes a human from the loop, but also enables
end-users to solve their problems in a few seconds as opposed to a few days
(while waiting for an expert's reply). 


\subsubsection{Supported SQL Subset}

The SQL specification contains over 1000 pages, and is impractical
to be fully used in practice, not alone major RDBMS vendors also provide
their own SQL dialects. In this work, we focus on a small subset
of the standard SQL 93 language. Such subset \textit{only} performs database
query operations but is expressive enough for most end-users' needs.

During our study of help forums, we also carefully studied what would
be the solution to users' desired query as they described to the experts
on the other side of the help forums. It turns out that the most common
desired queries can be composed by a handful of basic SQL constructs,
which form the target language of this work.



\subsubsection{Language Representation}

To bridge the gap between example-based specification and the result
SQL query, we need to identify an intermediate language representation
that is expressive enough to describe various query tasks succinctly,
while at the same time concise enough to be amenable for efficient
inference. 

It is worth noting that there is a tradeoff between the expressiveness
of a search space, and the complexity of finding simple consistent hypotheses
within that space. In general, the more expressive a search
space, the harder the task of finding consistent hypotheses within
that search space. However, it is also worth mentioning that the
expressiveness-complexity tradeoff is not as simple as it seems, as
an expressive language can sometimes make a simple theory fit the
input-output examples, whereas restricting the expressiveness of the
language representation means that any consistent theory must be very complex.

Using an intermediate language representation, rather than directly
translating examples to SQL syntax, for query synthesis has another benefit:
many RDBMS vendors provide their own SQL dialects, thus, performing
inference on the abstraction level of intermediate language and then
later instantiating the intermediate result
to a target SQL syntax offers more flexibility.


\subsubsection{Inference Algorithm}


We plan to take the recent advance in the research field of template-based
program synthesis~\cite{Gulwani:2010:DPS}, and
design an algorithm for synthesizing a desired SQL query
from input-output examples.

Concerning the outputs of the inference algorithm, we anticipate
that an expressive language representation for inductive
synthesis may result in a large number of programs (i.e., SQL
queries here) that satisfied the provided examples. 
We plan to address this problem by developing a ranking scheme that can be
used to rank the possibly large number of output results
that are consistent with a small number of input-output
examples. Using a ranking scheme is inspired by the
Occam�s razor principle, which states that a smaller and
simpler explanation is usually the correct one. Besides
selecting a smaller one, we could also compute the likelihood
of each query being correct based on some statistics, and then
pick the most likely one.



\subsubsection{User Interactive Model}
\label{sec:uim}

End-users can use our approach to obtain SQL query to transform
multiple, huge database tables by constructing small, representative
input and output example tables. On some examples, we speculate
that our approach
may produce a SQL query that satisfies the input and output examples
given by the user, but does not address the intention
that the user wants. To address this issue, we adapt a simple
interaction model from~\cite{Harris:2011} to ask users to investigate the results of
an output SQL query and report any discrepancy. In this case,
the user can refine the inferred SQL query by providing a more
informative input-output example (or multiple input-output examples
that together describe the required behavior) that demonstrate the behavior on
which the originally-inferred SQL query behaves incorrectly.

\subsection{Contributions}

We plan to make the following contributions in this work:

\begin{itemize}
\item \textbf{Technique.} An approach to automatically synthesizing SQL
queries from input-output examples (Section~\ref{sec:approach}).

\item \textbf{Tool.} A practical tool to implement the proposed approach (Section~\ref{sec:implementation}).

\item \textbf{Evaluation.} An empirical evaluation of the tool implementation
on SQL exercises in classic textbooks~\cite{cowbook}
and real-world questions end-users asked on online SQL help forums (Section~\ref{sec:evaluation}).
\end{itemize}
