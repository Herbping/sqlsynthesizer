
\section{Implementation and Evaluation}
\label{sec:evaluation}

We plan to implement the proposed approach in a programming
tool, and evaluate its practicability.

\subsection{Research Questions}

We aim to investigate the following research questions:

\begin{itemize}
\item Is the supported SQL subset expressive enough to describe
queries that real end-users require?

\item Can our inference algorithm efficiently find a SQL query
from input-output examples?

\item Does the inferred SQL query correctly express end-users'
intention and produce the expected output? If not, how many
additional examples does a user need to provide before our
approach infers a correctly-behaved SQL query?

\end{itemize}

\subsection{Evaluation Methodology}

We plan to perform the evaluation as follows to answer the
above research questions:

\begin{itemize}
\item \textbf{Subjects collection.} we have collected evaluation
subjects from two major sources: exercises from classic
database textbooks like~\cite{cowbook}, and real-world questions
that end-user asked on well-known online help forums. Specifically,
we aim to apply our implemented tool solve SQL query related questions
after Chapter 3 (SQL DML) in~\cite{cowbook}. We have also collected
19 end-user questions from online forums, and plan to apply our
tool to solve them.

\item{\textbf{Steps.}} For a given question, if that question
has already been associated with input-output examples (e.g.,
those from online forums), we directly apply our tool on the examples.
Otherwise, we manually provide a few concise and representative
examples for our tool.
If for a given problem, our tool inferred
a SQL query that does not behave as we expected when applied
to other inputs, we will manually find an input on which the
SQL query misbehaves and reapplied our tool to the new input. We
repeat this process until our tool infers a desirable SQL query.
\end{itemize}

Finally, we hope to illustrate the expressiveness of the supported SQL
subset language through experimenting on a diverse questions
people raised on the help-thread requests and standard
SQL questions a classic textbook would emphasize.

%%% Local Variables: 
%%% mode: latex
%%% TeX-master: "guierror"
%%% TeX-command-default: "PDF"
%%% End: 

%  LocalWords:  multithreaded SWT multithreading 2mm VirgoFTP UI GMail S3
%  LocalWords:  FileBunker ArecaBackup plugins HudsonEclipse EclipseRunner
%  LocalWords:  S3dropbox SudokuSolver Sudoku mutlithreaded SGTPuzzler 9pt
%  LocalWords:  Fennec MyTracks LOC v9306 plugin LOCC RTA CFA
%  LocalWords:  CalledByNativeMethods
