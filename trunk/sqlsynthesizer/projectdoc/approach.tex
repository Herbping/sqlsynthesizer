
\section{Approach}
\label{sec:approach}


\newcommand{\q}{\langle query\rangle}
\newcommand{\db}{\langle db\rangle}
\newcommand{\pat}{\langle pat\rangle}
\newcommand{\bug}{\langle bug\rangle}
\newcommand{\dist}{\langle distance\rangle}
\newcommand{\sem}[1]{\llbracket #1\rrbracket}
\newcommand{\lit}[1]{\texttt{#1}}

\newcommand{\column}{\langle column\rangle}
\newcommand{\dbtable}{\langle table\rangle}
\newcommand{\cond}{\langle cond\rangle}
\newcommand{\op}{\langle op\rangle}
\newcommand{\e}{\langle expr\rangle}
\newcommand{\ce}{\langle cexpr\rangle}

\begin{figure}[t]
%\scriptsize{%
\footnotesize%
\begin{align*}
\q ::= {} 
	& \texttt{ SELECT } \e^+ \texttt{ FROM } \dbtable^+ \\
        & \texttt{ WHERE } \cond^+ \\ 
	&  \texttt{ GROUP BY } \column^+ \texttt{ HAVING } \cond^+\\
\dbtable::= {} &\ atom \\
\column ::= {} &\ \dbtable.atom\\
\cond ::= {} &\ \ \cond \;\texttt{\&\&}\; \cond \\ 
    & |\ \cond \;\texttt{||}\; \cond \\
    & |\ \texttt{(}\;\cond\;\texttt{)} \\
    & |\ \ce \;\op\; \ce \\
\op ::= {} &\ \ \texttt{=} \;\;|\;\; \texttt{>}  \;\;|\;\; \texttt{<}\\
\ce ::= {} &\ \ const \;\;|\;\; \column  \;\; \\
\e ::= {} & \ce \\
    & |\ sum(\column) \;\;|\ max(\column) \;\;|\ min(\column) 
\end{align*}
\normalsize%
\caption{Syntax of the supported SQL language subset.}
\label{fig:syntax}
\end{figure}


\subsection{Language}

Figure~\ref{fig:syntax} defines the syntax of the supported
SQL language subset.

\sai{write more on this section. Why choose this language subset?}


\subsection{Inference Algorithm}

We describe an algorithm to infer good SQL queries from the given
input-output examples. Our algorithm consists of three steps:
table alignment (Section~\ref{sec:alignment}),
query completion (Section~\ref{sec:completion}), and
candidate ranking (Section~\ref{sec:ranking}).

\yuyin{can you please wrote the following 3 subsections? I wrote
down a few of my thoughts, which are only for reference. Feel free
to ignore them, and express your own ideas. }

\subsubsection{Table Alignment}
\label{sec:alignment}

Given an input-output example, this step infers a rough form of the
result SQL query. We may define some high-level templates to match
the given examples.

For example, from input-output examples, we can roughly know which
possible constructions (select, join, group by, or ordered by) will
be used. This step can reduce the overall search space.

\subsubsection{Query Completion}
\label{sec:completion}

After getting a rough idea of what the resulting SQL query would
look like, this step completes the remaining uncertain parts by
searching. Smarter search would be preferrable.

\subsubsection{Candidate Ranking}
\label{sec:ranking}

If a list of multiple SQL queries satisfy the given input-output examples, how
to present the most likely queries to the top?

