We next describe a real-world example picked up from
an online SQL help forum, that illustrates a typical scenario
in writing SQL queries\footnote{\url{http://forums.tutorialized.com/sql-basics-113/join-problem-147856.html}}.

The thread was started by a novice user, who needed help to write a
SQL query to get result from three input tables. In the help thread, the
novice user described his required query in a few paragraphs of
English, but also include several small, representative input-output
examples as shown in Figure~\ref{fig:example}, to better express
his intention. 

This thread recevies no replies as of May 2012, and we speculated that
joining three tables and then producing certain output result can
be non-trivial. We ran our tool on the exact same input/output examples
as shown in Figure~\ref{fig:example}. The tool produced 6 valid answers
in less than 1 minutes, all of which satisfy the given examples. The
highest ranked SQL query is shown in Figure~\ref{fig:answer}.
This query is unintuitive to write. It first needs to join three input tables
on columns \CodeIn{T1.Column2}, \CodeIn{T2.Column2},
\CodeIn{T2.Column1}, and \CodeIn{T3.Column1},
then filters out tuples whose \CodeIn{Table1.Column2} value
is not '2001'. Further, it aggregates the results based on
the value of \CodeIn{Table2.Column3}, and returns the minimal
values of columns \CodeIn{T1.Column1}, \CodeIn{T1.Column4}, and \CodeIn{T3.Column2}
in each aggregated group as the results.

%Such help threads on online discussion forums are not rare, but the SQL
%queries needed to address them are non-trivial.

\begin{figure}[t]

\textbf{The input Table1:}

\begin{tabular}{|c|c|c|c|}
\hline
 Column1 & Column2& Column3& Column4\\
 \hline
 101 &  2001 &  3020 &  01-01-11\\
\hline
 101 &  2001 &  3002 &  02-01-11\\
\hline
 101 &  2001 &  3001 &  03-01-11\\
\hline
 102 &  2002 &  3002 &  01-01-11\\
\hline
\end{tabular}

\vspace{2mm}

\textbf{The input Table2:}

\begin{tabular}{|c|c|c|}
\hline
 Column1 & Column2& Column3 \\
 \hline
 20011 &  2001 &  200131\\
\hline
 20012 &  2001 &  200132\\
\hline
 20013 &  2001 &  200133\\
\hline
\end{tabular}

\vspace{2mm}

\textbf{The input Table3:}

\begin{tabular}{|c|c|}
\hline
 Column1 & Column2\\
 \hline
 20011 &  Site\\
\hline
 20012 &  Site\\
\hline
 20013 &  Site\\
\hline
\end{tabular}

\vspace{3mm}

\textbf{The output table:}

\begin{tabular}{|c|c|c|c|}
\hline
 101 & 200131 & 01-01-11 & Site\\
 \hline
 101 & 200132 & 01-01-11 & Site\\
 \hline
 101 & 200133 & 01-01-11 & Site\\
 \hline
\end{tabular}
\Caption{{\label{fig:example} Input-output example tables from
an online SQL help forum thread. Our approach is expected to
automatically infer a SQL query as shown in Figure~\ref{fig:answer}
that produces the output table for the three input tables.
}} \vspace{2mm}
\end{figure}

\begin{figure}[t]
\begin{CodeOut}
\begin{alltt}
\textbf{select min(T1.Column1), T2.Column3,
       min(T1.Column4), min(T3.Column2)
from Table1 T1, Table2 T2, Table3 T3
where T1.Column2 = T2.Column2 and T2.Column1 = T3.Column1
       and T2.Column2 = '2001'
group by T2.Column3}
\end{alltt}
\end{CodeOut}
\vspace{-5mm}
\Caption{{\label{fig:answer} The expected SQL query inferred
by our approach for the input-output examples in Figure~\ref{fig:example}.}}
\end{figure}
