
\section{Related Work}
\label{sec:related}

Our work can be differentiated from existing work in three major ways: the scope
of the problem addressed, the technique used, and the evaluation methodology.
We next discuss two categories of closely-related work on reverse query processing
and automated program synthesis.


\subsection{Reverse Query Processing}

In the database community, there is a rich body of
work~\cite{Zloof:1975, Tran:2009, DasSarma:2010} that share the same broad principle
of ``reverse query processing''. Zloof's work on \textit{Query by Example} (QBE)
~\cite{Zloof:1975} provided an intuitive form-based Graphical User Interface for
writing database queries, which is completely different than ours.

The problem of \textit{Query By Output} (QBO)~\cite{Tran:2009} shares a similar
goal with QBE, but takes a quite different approach.
In QBO, we are given a view: $V$ and $n$ relations: $R_1, R_2, ..., R_n$ and asked to
return a query $Q$ such that $Q(R_1, R_2, ..., R_n)$ = $V$. However, QBO is cast as
a classification problem (into two classes: tuples in $V$, and tuples not in $V$)
and decision trees are used to find a compact query. A decision tree is constructed
in a greedy fashion by determining a ``good'' predicate to split the tuples into two
classes, which form the root nodes of two decisions trees (each of which is
constructed in a recursive fashion).  The QBO technique presented in~\cite{Tran:2009}
infers select-project-join queries. Such queries can only be applied to
satisfy a limited number of examples, and cannot be applied to many of the other
real world cases that we studied (e.g., a query with an aggregation operator
like our motivating example in Figure~\ref{fig:example}).

Recently, the \textit{View Definitions Problem} (VDP) has been addressed
by Sarma et al.~\cite{DasSarma:2010}.
VDP is restricted to find the most succinct and accurate view definition, when
the view query is restricted to a specific family of queries. VDP is the special
case of QBO where $n = 1$ and there are no joins nor projections.
The high-level goal of this work is similar to our approach, but there
are some key differences: (1) VDP techniques infer a relation from a large representative
example view, while we infer a SQL query from a set of few input-output examples
(which is a critical usability aspect for end-users who lack adequate
SQL experience). (2) the VDP technique in~\cite{DasSarma:2010} does
not consider join or projection
operations and assumes the output view to be a strict subset of the input
database with the same schema, while our work considers both join and
projection operations.


In the data mining community, a slightly related problem to ours is
the \textit{redescription mining} problem~\cite{Ramakrishnan:2004}.
At an abstract level, our work is different from these redescription mining
techniques in several ways.
First, the goal of redescription mining is to find different subsets of
data that afford multiple descriptions across multiple vocabularies
covering the same dataset, while our goal is to infer a SQL statement
to obtain the desirable querying behavior.
Second, we are concerned with a fixed subset of the data (i.e., the output example table).
Third, none of the existing approaches for redescription mining account for
structural (relational) information in the data. 

Other query recommendation systems proposed in the literature may achieve
a similar goal of helping end-users write better SQL queries, but they rely on information
that we cannot assume access to in many realistic scenarios: a query log~\cite{Khoussainova:2010},
or user history and preferences~\cite{Howe:2011}.

\subsection{Automated Program Synthesis }

Program synthesis is the problem of creating a program
in some underlying language from specifications that can range
from logical declarative specifications to examples or
demonstrations~\cite{Harris:2011, singh:2012, Gulwani:2011}.
This topic has been studied intensively in Artificial Intelligence and
Programming Language communities with the goal of easing the burden of
algorithm designers, software developers, or end-users.
Interest in it has grown rapidly in recent years~\cite{Gulwani:2010:DPS},
and many approaches have been proposed.

Example-based program synthesis techniques~\cite{Kandel:2011, Fisher:2008,
Fisher08Pads,Lau:2003:PDU, Lau:2000:VSA, Barbosa:2010:MLA, Arasu:2009:LST}
take textual input-output examples and infer programs that transform text.
However, existing techniques focused on synthesizing programs for string
expressions~\cite{singh:2012, Gulwani:2011}, spreadsheet layout transformation~\cite{Harris:2011},
, geometry constructions~\cite{Gulwani:2011:SGC}, thus
can not be directly applied to our SQL query synthesis domain,
due to different abstractions, granularities, and tasks.

