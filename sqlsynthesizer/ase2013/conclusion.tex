\section{Conclusions and Future Work}
\label{sec:conclusion}

\todo{move to somewhere else}
General purpose programming platforms have never been easy for
end-users who are not professional programmers.
Users are often stucked with the process
of \textit{how} to accomplish a certain task by receiving 
step-by-step, detailed, and syntactically correct instructions, 
instead of simply describing
\textit{what} the task is. 

%\textit{Programming by examples} has the potential
%to change this landscape, when targeted for the right set of people for the
%right set of problems.

This paper studied the problem of automated SQL query synthesis
from simple input-output examples, and presented a
practical technique (and its tool implementation, called \ourtool)
that can infer desirable SQL queries that fulfill end-users' intentions.
Our experimental results show that \ourtool is
effective in automating a variety of database query tasks,
and it does so with small input-output examples.

The source code of our tool implementation is available at: \\
\url{http://sqlsynthesizer.googlecode.com}

\vspace{1mm}

Our future work will concentrate on the following topics:

\textbf{Enrich the supported SQL subset.} \ourtool supports
a SQL subset containing some widely-used language features.
\todo{miss some corner cases}
We plan to enrich the
supported SQL subset, and design a corresponding algorithm
to synthesize more general queries.

\textbf{Illustration of synthesis steps.} Besides
producing a final result, end-users may also be interested in knowing
how a SQL query is inferred step by step.
Showing detailed inference steps not
only makes \ourtool more usable, but also permits
end-users to better understand the whole process and
spot possible errors earlier.
We plan to apply recent advance in data visualization~\cite{Kandel:2011}
to the context of program synthesis.

\textbf{Noise detection and tolerance in users' inputs.} The current technique
requires users to provide noise-free input-output examples.
Even in the presence of a small amount of user-input noises (e.g., a typo),
the inference algorithm will declare failure when it fails to learn
a valid SQL query.
To overcome this limitation, we plan to design a more robust inference
algorithm that can attempt to identify and tolerate user-input noises,
and even suggest a fix to the noisy example.



%\textbf{Generalization.}
%We have applied our synthesis technique to one specific, but
%important problem.  There is a rich history of end-user programming
%problems being cast as program synthesis (e.g.,~\cite{Gulwani:2010:DPS}).  In future
%work, we plan to formulate other tasks, such as repetitive file operations, and to apply our technique to them.

