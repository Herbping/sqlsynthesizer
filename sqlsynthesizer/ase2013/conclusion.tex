\section{Conclusion and Future Work}
\label{sec:conclusion}

%\todo{move to somewhere else}


%\textit{Programming by examples} has the potential
%to change this landscape, when targeted for the right set of people for the
%right set of problems.

This paper studied the problem of automated SQL query synthesis
from simple input-output examples, and presented a
practical technique (and its tool implementation, called \ourtool).
\ourtool is motivated by the growing population of
non-expert database users, who need to query on their
databases, but have difficulty with SQL.
We have shown that that \ourtool is
able to synthesize a variety of SQL queries,
and it does so with small input-output examples. We view
\ourtool as an important step toward making databases
more usable.
The source code of our tool implementation is available at: 
\url{http://sqlsynthesizer.googlecode.com}


For future work, we are interested in conducting a user study
to evaluate \ourtool's usability. We also plan to explore
applications of this technique.
%enrich the
%supported SQL subset for sythesizing more expressive queries.

\vspace{1mm}

\begin{comment}
Our future work will concentrate on the following topics:

\textbf{Enrich the supported SQL subset.} We plan to enrich the
supported SQL subset by \ourtool, and design a corresponding algorithm
to synthesize more general queries.

\textbf{Illustration of synthesis steps.} Besides
producing a final result, end-users may also be interested in knowing
how a SQL query is inferred step by step.
Showing detailed inference steps not
only makes \ourtool more usable, but also permits
end-users to better understand the whole process and
spot possible errors earlier.
We plan to apply recent advance in data visualization~\cite{Kandel:2011}
to the context of program synthesis.

\textbf{Noise detection and tolerance in users' inputs.} The current technique
requires users to provide noise-free input-output examples.
Even in the presence of a small amount of user-input noises (e.g., a typo),
the inference algorithm will declare failure when it fails to learn
a valid SQL query.
To overcome this limitation, we plan to design a more robust inference
algorithm that can attempt to identify and tolerate user-input noises,
and even suggest a fix to the noisy example.

\end{comment}


