\begin{abstract}

Many computer end-users, such as
research scientists and business analysts, need to frequently query
a database, yet lack the programming knowledge to do such tasks smoothly.
To alleviate this problem, we present
a \textit{programming by example} technique
(and its tool implementation, called \ourtool)
that permits end-users to automate such querying tasks.
Our technique takes from users an input and output example of how the
database should be queried, and then synthesizes a SQL query that
reproduces the example output from the example input.
Later, when the synthesized
SQL query is applied to another, potentially larger, database with a
similar schema as the example input, the synthesized SQL query produces
a corresponding result that is similar to the example output. 

\todo{may delay this paragraph later}
Our technique
has several notable features: it only needs small input-output examples
to infer a desirable SQL query, it is fully automated and does not require
users to provide annotations/hints of any form, and it can rank multiple
possible solutions to provide to users the most likely result.

\todo{num is missing here}
We evaluated \ourtool on \exnum SQL exercises from a classic
database textbook and \pnum non-trivial SQL questions raised by
real-world users from popular online forums.
As a result, \ourtool infers correct SQL queries for 
\solexnum textbook exericses and \solpnum forum questions,
including those that receive no human replies.

%Our technique has been implemented as an open-source programming
%tool. In our preliminary evaluation, our prototype tool has synthesized correct
%answers for 5 out of 6 SQL exercises from a classic database textbook,
%and has been used to solve 5 non-trivial problems raised by real-world
%users from popular online forums, 

%This opens up an amazing
%set of possibilities in the context of making classroom teaching
%interactive


\end{abstract}
