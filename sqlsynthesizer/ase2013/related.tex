
\section{Related Work}
\label{sec:related}

This section discusses two categories
of closely-related work on on reverse query processing methods
and automated program synthesis techniques.


\subsection{Reverse Query Processing}

In the database community, there is a rich body of
work~\cite{Zloof:1975, Tran:2009, DasSarma:2010} that share the same broad principle
of ``reverse query processing''. Zloof's work on \textit{Query by Example} (QBE)
~\cite{Zloof:1975} provided an intuitive form-based Graphical User Interface for
writing database queries, which is completely different than ours.
\todo{add why it is different}

\todo{mention the difference} Tran et al.~\cite{Tran:2009} studied the
problem of \textit{Query By Output} (QBO). QBO shares a similar
goal with QBE, but is significantly different than \ourtool.
In QBO, we are given a view: $V$ and $n$ relations: $R_1, R_2, ..., R_n$ and asked to
return a query $Q$ such that $Q(R_1, R_2, ..., R_n)$ = $V$. 
\todo{what's the difference between?}
However, QBO is cast as
a classification problem (into two classes: tuples in $V$, and tuples not in $V$)
and decision trees are used to find a compact query. A decision tree is constructed
in a greedy fashion by determining a ``good'' predicate to split the tuples into two
classes, which form the root nodes of two decisions trees (each of which is
constructed in a recursive fashion).  The QBO technique presented in~\cite{Tran:2009}
infers select-project-join queries. Such queries can only be applied to
satisfy a limited number of examples, and cannot be applied to many of the other
real world cases that we studied (e.g., queries with aggregation operators
such the examples in Figure~\ref{fig:motivating} and~\ref{fig:example2}).

Recently, Sarma et al.~\cite{DasSarma:2010} proposed a technique
to address the \textit{View Definitions Problem} (VDP).
\todo{explain more here}
With a rather different goal than \ourtool,
VDP aims to to find the most succinct and accurate view definition, when
the view query is restricted to a specific family of queries.
VDP can be treated as a special
case of QBO where $n = 1$ and there are no joins nor projections.
Compared to \ourtool, there are three key differences:
(1) VDP aims to infer a relation from a large representative
example view, while \ourtool attempts to infer a SQL query from a
set of few input-output examples \todo{revise below}
(which is a critical usability aspect for end-users who lack adequate
SQL experience). (2) the technique proposed by Sarma et al.~\cite{DasSarma:2010} does
not consider join or projection
operations and assumes the output view to be a strict subset of the input
database with the same schema\todo{revise}, while our work eliminates these
assumptions, and considers both join and projection operations.\todo{revise above}


In the data mining community, a slightly related problem,
called the \textit{redescription mining problem}~\cite{Ramakrishnan:2004},
has been recently studied by researchers.
\todo{explain the problem}
However, \ourtool is significantly
%ours is the \textit{redescription mining} problem~\cite{Ramakrishnan:2004}.
%At an abstract level, our work is 
different from existing redescription mining techniques~\cite{} in several ways.
First, the goal of redescription mining is to find different subsets of
data that afford multiple descriptions across multiple vocabularies
covering the same dataset, while our goal is to infer a SQL query 
to achieve the desirable output. 
Second, we are concerned with a fixed subset of the data (i.e., the output example table).
\todo{explain}
Third, none of the existing redescription mining techniques account for
structural (relational) information in the data. \todo{revise}

\todo{what is the limitation of non program sythesis, why
sythensis would be useful.}

\subsection{Automated Program Synthesis }



Program synthesis aims to create an executable program
in some underlying language from specifications that can range
from logical declarative specifications to examples or
demonstrations~\cite{Harris:2011, singh:2012, Gulwani:2011}.
This topic has been studied intensively in Artificial Intelligence and
Programming Language communities with the goal of easing the burden of
algorithm designers, software developers, or end-users.
Interest in it has grown rapidly in recent years~\cite{Gulwani:2010:DPS},
and many approaches have been proposed.
\todo{need to inspect}
Closely-related to \ourtool, example-based program synthesis
techniques~\cite{Kandel:2011, Fisher:2008,
Fisher08Pads,Lau:2003:PDU, Lau:2000:VSA, Barbosa:2010:MLA, Arasu:2009:LST}
take textual input-output examples and infer programs that transform text.
However, existing techniques focused on synthesizing programs for string
expressions~\cite{singh:2012, Gulwani:2011}, spreadsheet layout transformation~\cite{Harris:2011},
, geometry constructions~\cite{Gulwani:2011:SGC}, thus
can not be directly applied to our SQL query synthesis domain,
due to different abstractions, granularities, and tasks.
\todo{explain the difference between excel work}

There is few work for inferring SQL queries for database end-users.
Query recommendation systems, such as SQLShare~\cite{} and SQLSnippet~\cite{},
\todo{explain}
may achieve a similar goal of helping end-users write better SQL
queries, but they rely on information
that we cannot assume access to in many realistic scenarios: a query log~\cite{Khoussainova:2010},
or user history and preferences~\cite{Howe:2011}.
\todo{inferring SQL from code, for performance}
Add this as related work~\cite{abs-1208-2013}

%also cite this~\cite{Cheung:2012:UPS}, combining machine learning with program synthesis.
